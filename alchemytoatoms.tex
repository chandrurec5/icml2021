%%%%%%%% ICML 2021 EXAMPLE LATEX SUBMISSION FILE %%%%%%%%%%%%%%%%%

\documentclass{article}

% Recommended, but optional, packages for figures and better typesetting:
\usepackage{microtype}
\usepackage{graphicx}
\usepackage{subfigure}
\usepackage{booktabs} % for professional tables

% hyperref makes hyperlinks in the resulting PDF.
% If your build breaks (sometimes temporarily if a hyperlink spans a page)
% please comment out the following usepackage line and replace
% \usepackage{icml2021} with \usepackage[nohyperref]{icml2021} above.
\usepackage{hyperref}
\usepackage{cleveref}

% Attempt to make hyperref and algorithmic work together better:
\newcommand{\theHalgorithm}{\arabic{algorithm}}

% Use the following line for the initial blind version submitted for review:
\usepackage{icml2021}

% If accepted, instead use the following line for the camera-ready submission:
%\usepackage[accepted]{icml2021}```

% The \icmltitle you define below is probably too long as a header.
% Therefore, a short form for the running title is supplied here:
\icmltitlerunning{Submission and Formatting Instructions for ICML 2021}

\begin{document}

\twocolumn[
\icmltitle{Deep Neural Networks are both layers and paths : From Alchemy To Atoms}

% It is OKAY to include author information, even for blind
% submissions: the style file will automatically remove it for you
% unless you've provided the [accepted] option to the icml2021
% package.

% List of affiliations: The first argument should be a (short)
% identifier you will use later to specify author affiliations
% Academic affiliations should list Department, University, City, Region, Country
% Industry affiliations should list Company, City, Region, Country

% You can specify symbols, otherwise they are numbered in order.
% Ideally, you should not use this facility. Affiliations will be numbered
% in order of appearance and this is the preferred way.
\icmlsetsymbol{equal}{*}

\begin{icmlauthorlist}
\icmlauthor{Chandrashekar}{s}`
\end{icmlauthorlist}

\icmlaffiliation{to}{Department of Computation, University of Torontoland, Torontoland, Canada}
\icmlaffiliation{goo}{Googol ShallowMind, New London, Michigan, USA}
\icmlaffiliation{ed}{School of Computation, University of Edenborrow, Edenborrow, United Kingdom}

\icmlcorrespondingauthor{Cieua Vvvvv}{c.vvvvv@googol.com}
\icmlcorrespondingauthor{Eee Pppp}{ep@eden.co.uk}

% You may provide any keywords that you
% find helpful for describing your paper; these are used to populate
% the "keywords" metadata in the PDF but will not be shown in the document
\icmlkeywords{Machine Learning, ICML}

\vskip 0.3in
]

% this must go after the closing bracket ] following \twocolumn[ ...

% This command actually creates the footnote in the first column
% listing the affiliations and the copyright notice.
% The command takes one argument, which is text to display at the start of the footnote.
% The \icmlEqualContribution command is standard text for equal contribution.
% Remove it (just {}) if you do not need this facility.

%\printAffiliationsAndNotice{}  % leave blank if no need to mention equal contribution
\printAffiliationsAndNotice{\icmlEqualContribution} % otherwise use the standard text.

\section{Introduction}\label{sec:intro}
An important question plaguing machine learning is:
\begin{center}
\textbf{\emph{Is Deep Learning Alchemy?}}
\end{center}
The \emph{alchemy} question can further be broken down into sub-questions as listed below. 

1. \emph{Training:} Despite a \emph{non-convex} loss surface, why is it possible for stochastic gradient descent (SGD) (and its variants) to achieve zero training error in standard deep neural networks (DNNs)?

2. \emph{Generalisation:} Standard DNNs are \emph{over-parameterised}, yet, when trained on data with true labels we observe good test performance. Does understanding deep learning require rethinking generalisation? \cite{ben}

3. \emph{Depth:} Approximation results show that more depth is better \cite{depth1,depth2}, i.e., deeper models can approximate more complicated target functions better than shallow models. Yet, when training on standard datasets increasing the depth beyond a point adversely affects both training and test performance \cite{resnets}. This situation can be remedied by using \emph{skip connections} giving rise to residual neural networks. However, why increasing depth beyond a point hurts standard DNNs is still not understood satisfactorily. 

4. \emph{Depth vs Width:} Wider models \cite{wide1,wide2,wide3} have also been quite successful. While increasing either depth or width causes an increase in the number of model parameters, the trade-off between width and depth is not clearly understood.

5. \emph{Functionality:} The roles of the basic parts namely weights and activation have not been clearly understood.

6. \emph{Interpretability of Representation:} The commonly held view of feature learning is that lower level features are learnt in the initial layers and as one proceeds in depth more sophisticated features are learnt in the higher levels and the final layer learns a linear model in the features given by the output of the penultimate layer. However, there is no straightforward way to interpret the features learnt in the hidden layers.

As the alchemy question is being debated  \cite{BenAli-1,Lecun,BenAli-2,Aliresponse,Mickens} in the machine learning community, side by side theoretical as well as empirical efforts are begin made to obtain useful insights to understand DNNs. Common recurring themes in such works are (i) \emph{simplification} procedures such as pruning which involves systemically throwing away unnecessary weights and (ii) \emph{extremisation} approaches such as studying the behaviour of DNNs say in the limit of infinite width or say in the presence of random labels. Many a times such simplification/extremisation has resulted in more intriguing results and thrown further open questions. We will now describe some of these.

{Pruning} techniques have been known to reduce  the size of DNNs upto even $90\%$ without significant loss in performance \cite{prune1,prune2,prune3,prune4}. \cite{lottery} made an interesting observation that smaller networks pruned network can be re-trained to match the performance of the original unpruned network only when the pruned network is trained starting from its original initialisation. This lead to the so called \emph{lottery ticket} hypothesis, i.e., ``a randomly initialised, dense neural network contains a sub network that is initialised such that—when trained in isolation—it can match the test accuracy of the original network after training for at most the same number of iterations".

Training with random labels has been an extremisation approach. \cite{ben} showed that standard DNNs can achieve zero training error even when fitting random/shuffled labels, shuffled/random pixels on standard datasets. However, when trained on data with true uncorrupted labels, they achieve good test performance. In a recent paper, \cite{randlabel} showed pre-training with random labels could result in both \emph{positive} as well as \emph{negative} effect on the speed of downstream re-training with true labels depending on 	factors such as initialisation scale and number of random classes upstream. However, \cite{randlabel} also observe that the test performance of the downstream model always degrades when the model is pre-trained with random labels. It is an open question to understand this adverse effect on test performance.

Another extremisation approach is to study DNNs in the limit of infinite width. \cite{fcgp,convgp,arora2019exact,cao2019generalization} have related kernel methods to infinite width DNNs. Two important kernels associated with a (finite as well as infinite width) DNN are its \emph{conjugate kernel} derived from the output of the DNN and the so called \emph{neural tangent kernel} based on the gradient of the output with respect to the DNN weights. As width approaches to infinity, both the conjugate and the neural tangent kernel converge to (their corresponding) deterministic matrices. It has been shown that training the last layer of the infinite width DNN is equivalent to a kernel method with the deterministic conjugate kernel and training  all the layers of an infinite width DNN is equivalent to a kernel method with the deterministic neural tangent kernel. Thus training and generalisation of infinite width DNNs boils down to the properties of the limiting deterministic kernel matrix. On a standard dataset such as CIFAR-10 the neural tangent kernel performs better than the conjugate kernel as well as other prior pure kernel based methods. However, standard finite width convolutional neural network still outperforms its infinite width neural tangent kernel counterpart by $5-6\%$. 

Even though the neural tangent kernel settles the case of infinite width DNNs, there are some interesting open questions. Firstly, we need to understand why  finite width neural networks outperform their infinite width neural tangent kernel counterparts. Secondly, feature learning is believed to be the unique differentiator of DNNs and other the machine learning methods, and neural tangent kernel being a kernel method no feature learning.



\textbf{Our Goal:} The goal of this paper is to champion a dual view as a pedagogical nugget of ``simple theory and simple experiments'' \cite{Aliresponse} to obtain insights about `practical' deep neural networks (DNNs). By `practical', we mean DNNs with standard architectural choices (such as fully connected, convolutional, pooling layers, use of skip connections) trained using variants of stochastic gradient descent (SGD) starting from any of the widely used randomised initialisations. By `practical', we also mean to exclude theory that addresses solely approximation/capacity related questions \cite{depth1,depth2}.

The primal/dual view is succinctly put as below. 
\begin{center}
\begin{tabular}{ccl}
\emph{Primal View}&:& Layer by layer.\\
\emph{Dual View}&:& Path by path.\\
\end{tabular}
\end{center}

Standard DNNs are \emph{over-parameterised}, so much so that, they 


\bibliographystyle{plainnat}
\bibliography{refs}
%%\section{Neural Path Framework For CNN}\label{sec:cnpf}
\textbf{Indexing:} The weights of layers $l\in[\dc]$ are denoted by $\Theta(\icin,\iin,\iout,l)$ and for layers $l\in[\dfc]+\dc$ are denoted by $\Theta(\iin,\iout,l)$. The pre-activations, gating and hidden unit outputs are denoted by $q_{x,\Theta}(\ifout,\iout,l)$,  $G_{x,\Theta}(\ifout,\iout,l)$, and $z_{x,\Theta}(\ifout,\iout,l)$ for layers $l=1,\ldots, \dc$. $\iin$ and $\iout$ are used to index the input and the output filters. $\ifout$ is used to denote the index of hidden unit (in the feature dimension) within the input and output filters. %Here, $\icin\in[\wconv]$, for $l=1\ldots,\dc$, $\iin\in[w]$ for $l=\dc+2,\ldots,\dc+\dfc+1$, $\ifin \in[1]$ for $l=1$, $\ifin \in[w]$ for $l=2,\ldots,\dc$, $\iout\in[w]$ for $l=1\ldots,\dc, \dc+2,\ldots, \dc+\dfc$,  $\iout\in[1]$ for $l=\dc+\dfc+1$, $\ifout\in[w]$ for $l=1,\ldots,\dc$.
\begin{comment}
\FloatBarrier
\begin{table}[h]
\centering
\begin{tabular}{|c|ll|}\hline
Index & Range&\\\hline
\multirow{2}{*}{$\iin$} & $\in[\din]$ & for $l=1\ldots,\dc$\\ \cline{2-3}
&$\in[w]$ & for $l=\dc+2,\ldots,\dc+\dfc+1$\\ \hline
\multirow{2}{*}{$\ifin$} & $\in[1]$ & for $l=1$\\ \cline{2-3}
&$\in[w]$ &for $l=2,\ldots,\dc$\\ \hline
\multirow{2}{*}{$\iout$} & $\in[w]$ &for $l=1\ldots,\dc, \dc+2,\ldots, \dc+\dfc$\\ \cline{2-3}
&$\in[1]$ &for $l=\dc+\dfc+1$\\ \hline
{$\ifout$} & $\in[w]$ &for $l=1,\ldots,\dc$\\ \hline
\end{tabular}
\end{table}
\end{comment}

\textbf{Shapes:} \Cref{fig:shape-main} shows the shapes of the tensors in the convolutional layers of a $1$-dimensional circular CNN considered in this paper. Here, the input is a $1$-dimensional tensor given by $x\in\R^{\din}$. The hidden nodes in a given convolutional layer have a $2$-dimensional shape of $\din\times w$, where $w$ is the number of filters in the layer. The weights of a given convolutional layer have $3$-dimensional shape of $\wconv\times w\times w$,  where $w\times w$ is because of the number of input filters times the number of output filters.
\FloatBarrier
\begin{figure}[h]
\centering
%\resizebox{\columnwidth}{!}{
\includegraphics[scale=0.04]{figs/shape.png}
%}
\label{fig:shape-main}
\caption{Shows the shape of the tensor.}
\end{figure}

\subsubsection{Information Flow}
\begin{table}[h]
\centering
\begin{tabular}{|c l lll|}\hline
IL&: &$z_{x,\Theta}(\cdot,1,0)$ &$=$ &$x$ \\\hline\hline
\multicolumn{5}{|l|}{Convolutional Layers, $l\in[\dc]$}\\\hline\hline
PA&: & $q_{x,\Theta}(\ifout,\iout,l)$& $=$ & $\sum_{\icin,\iin}\Theta(\icin,\iin,\iout,l)\cdot z_{x,\Theta}(\ifout\oplus (\icin-1),\iin,l-1)$\\
GV&: &$G_{x,\Theta}(\ifout,\iout,l)$& $=$ & $\mathbf{1}_{\{q_{x,\Theta}(\ifout,\iout,l)>0\}}$\\
HUO&: &$z_{x,\Theta}(\ifout,\iout,l)$ & $=$ & $q_{x,\Theta}(\ifout,\iout,l)\cdot G_{x,\Theta}(\ifout,\iout,l)$\\\hline\hline
\multicolumn{5}{|l|}{GAP Layers, $l=\dc+1$}\\\hline\hline
%HUO&: &${z}_{x,\Theta}(\iout,l)$ & $=$ & $\frac{1}{\din}\sum_{i\in [\din]} z_{x,\Theta}(i,\iout,l-1)$\\\hline\hline
HUO&: &$z_{x,\Theta}(\iout, \dc+1)$ & $=$ &$\sum_{\ifout} z_{x,\Theta}(\ifout,\iout,\dc)\cdot G^{\text{pool}}_{x,\Theta}(\ifout,\iout,\dc+1)$\\\hline\hline
\multicolumn{5}{|l|}{Fully Connected Layers, $l\in[\dfc]+(\dc+1)$}\\\hline\hline
PA&: & $q_{x,\Theta}(\iout,l)$& $=$ & $\sum_{\iin}\Theta(\iin,\iout,l) \cdot z_{x,\Theta}(\iin,l-1) $\\
GV&: &$G_{x,\Theta}(\iout,l)$& $=$ & $\mathbf{1}_{\{(q_{x,\Theta}(\iout,l))>0\}}$\\
HUO&: &$z_{x,\Theta}(\iout,l)$ & $=$ & $q_{x,\Theta}(\iout,l)\cdot G_{x,\Theta}(\iout,l)$\\
FO&: & $\hat{y}_{\Theta}(x)$ & $=$ & $\sum_{\iin}\Theta(\iin,\iout, d)\cdot z_{x,\Theta}(\iin,d-1)$\\\hline
\end{tabular}
\caption{Here IL, PA, GV, HUO, GL and FO are abbreviations for input layer, pre-activation, gating values, hidden unit output, GAP-layer and final output respectively.}
\label{tb:cconv}
\end{table}

\FloatBarrier
\begin{figure}[H]
\centering
\resizebox{\columnwidth}{!}{
\includegraphics[scale=1]{figs/single-filter.png}
}
\end{figure}
\newpage

%%\input{supp}
%\newpage
\onecolumn
\begin{center}
{\Large{\textbf{Appendix}}}
\end{center}

\begin{appendix}
\section{Neural Path Framework For CNN}\label{sec:cnpf}
\textbf{Indexing:} The weights of layers $l\in[\dc]$ are denoted by $\Theta(\icin,\iin,\iout,l)$ and for layers $l\in[\dfc]+\dc$ are denoted by $\Theta(\iin,\iout,l)$. The pre-activations, gating and hidden unit outputs are denoted by $q_{x,\Theta}(\ifout,\iout,l)$,  $G_{x,\Theta}(\ifout,\iout,l)$, and $z_{x,\Theta}(\ifout,\iout,l)$ for layers $l=1,\ldots, \dc$. $\iin$ and $\iout$ are used to index the input and the output filters. $\ifout$ is used to denote the index of hidden unit (in the feature dimension) within the input and output filters. %Here, $\icin\in[\wconv]$, for $l=1\ldots,\dc$, $\iin\in[w]$ for $l=\dc+2,\ldots,\dc+\dfc+1$, $\ifin \in[1]$ for $l=1$, $\ifin \in[w]$ for $l=2,\ldots,\dc$, $\iout\in[w]$ for $l=1\ldots,\dc, \dc+2,\ldots, \dc+\dfc$,  $\iout\in[1]$ for $l=\dc+\dfc+1$, $\ifout\in[w]$ for $l=1,\ldots,\dc$.
\begin{comment}
\FloatBarrier
\begin{table}[h]
\centering
\begin{tabular}{|c|ll|}\hline
Index & Range&\\\hline
\multirow{2}{*}{$\iin$} & $\in[\din]$ & for $l=1\ldots,\dc$\\ \cline{2-3}
&$\in[w]$ & for $l=\dc+2,\ldots,\dc+\dfc+1$\\ \hline
\multirow{2}{*}{$\ifin$} & $\in[1]$ & for $l=1$\\ \cline{2-3}
&$\in[w]$ &for $l=2,\ldots,\dc$\\ \hline
\multirow{2}{*}{$\iout$} & $\in[w]$ &for $l=1\ldots,\dc, \dc+2,\ldots, \dc+\dfc$\\ \cline{2-3}
&$\in[1]$ &for $l=\dc+\dfc+1$\\ \hline
{$\ifout$} & $\in[w]$ &for $l=1,\ldots,\dc$\\ \hline
\end{tabular}
\end{table}
\end{comment}

\textbf{Shapes:} \Cref{fig:shape-main} shows the shapes of the tensors in the convolutional layers of a $1$-dimensional circular CNN considered in this paper. Here, the input is a $1$-dimensional tensor given by $x\in\R^{\din}$. The hidden nodes in a given convolutional layer have a $2$-dimensional shape of $\din\times w$, where $w$ is the number of filters in the layer. The weights of a given convolutional layer have $3$-dimensional shape of $\wconv\times w\times w$,  where $w\times w$ is because of the number of input filters times the number of output filters.
\FloatBarrier
\begin{figure}[h]
\centering
%\resizebox{\columnwidth}{!}{
\includegraphics[scale=0.04]{figs/shape.png}
%}
\label{fig:shape-main}
\caption{Shows the shape of the tensor.}
\end{figure}

\subsubsection{Information Flow}
\begin{table}[h]
\centering
\begin{tabular}{|c l lll|}\hline
IL&: &$z_{x,\Theta}(\cdot,1,0)$ &$=$ &$x$ \\\hline\hline
\multicolumn{5}{|l|}{Convolutional Layers, $l\in[\dc]$}\\\hline\hline
PA&: & $q_{x,\Theta}(\ifout,\iout,l)$& $=$ & $\sum_{\icin,\iin}\Theta(\icin,\iin,\iout,l)\cdot z_{x,\Theta}(\ifout\oplus (\icin-1),\iin,l-1)$\\
GV&: &$G_{x,\Theta}(\ifout,\iout,l)$& $=$ & $\mathbf{1}_{\{q_{x,\Theta}(\ifout,\iout,l)>0\}}$\\
HUO&: &$z_{x,\Theta}(\ifout,\iout,l)$ & $=$ & $q_{x,\Theta}(\ifout,\iout,l)\cdot G_{x,\Theta}(\ifout,\iout,l)$\\\hline\hline
\multicolumn{5}{|l|}{GAP Layers, $l=\dc+1$}\\\hline\hline
%HUO&: &${z}_{x,\Theta}(\iout,l)$ & $=$ & $\frac{1}{\din}\sum_{i\in [\din]} z_{x,\Theta}(i,\iout,l-1)$\\\hline\hline
HUO&: &$z_{x,\Theta}(\iout, \dc+1)$ & $=$ &$\sum_{\ifout} z_{x,\Theta}(\ifout,\iout,\dc)\cdot G^{\text{pool}}_{x,\Theta}(\ifout,\iout,\dc+1)$\\\hline\hline
\multicolumn{5}{|l|}{Fully Connected Layers, $l\in[\dfc]+(\dc+1)$}\\\hline\hline
PA&: & $q_{x,\Theta}(\iout,l)$& $=$ & $\sum_{\iin}\Theta(\iin,\iout,l) \cdot z_{x,\Theta}(\iin,l-1) $\\
GV&: &$G_{x,\Theta}(\iout,l)$& $=$ & $\mathbf{1}_{\{(q_{x,\Theta}(\iout,l))>0\}}$\\
HUO&: &$z_{x,\Theta}(\iout,l)$ & $=$ & $q_{x,\Theta}(\iout,l)\cdot G_{x,\Theta}(\iout,l)$\\
FO&: & $\hat{y}_{\Theta}(x)$ & $=$ & $\sum_{\iin}\Theta(\iin,\iout, d)\cdot z_{x,\Theta}(\iin,d-1)$\\\hline
\end{tabular}
\caption{Here IL, PA, GV, HUO, GL and FO are abbreviations for input layer, pre-activation, gating values, hidden unit output, GAP-layer and final output respectively.}
\label{tb:cconv}
\end{table}

\FloatBarrier
\begin{figure}[H]
\centering
\resizebox{\columnwidth}{!}{
\includegraphics[scale=1]{figs/single-filter.png}
}
\end{figure}
\newpage

\section{Proofs of technical results}

Proof of \Cref{prop:bundle}
\begin{proof}
Note that the total number of paths is $P=\din\cdot (\wconv\cdot w)^{\dc} \cdot \wfc^{(\dfc-1)}$, and in the definition of NPV for CNNs in \Cref{def:bundle} the indices are over only the weights without specifying the input node given by $\Ifeat_{0}(p)$.
\end{proof}

\begin{proposition}[Rotational Invariance]\label{prop:rot}
Internal variables in the convolutional layers are circularly symmetric,  i.e., for $r\in\{0,\ldots,\din-1\}$ it follows that (i) $z_{rot(x,r),\Theta}(\ifout,\cdot,\cdot) = z_{x,\Theta}(\ifout \oplus r,\cdot,\cdot)$, (ii) $q_{rot(x,r),\Theta}(\ifout,\cdot,\cdot) = q_{x,\Theta}(\ifout \oplus r,\cdot,\cdot)$ and (iii) $G_{rot(x,r),\Theta}(\ifout,\cdot,\cdot) = G_{x,\Theta}(\ifout \oplus r,\cdot,\cdot)$.
\end{proposition}


\begin{proof}
For $l=0$, we have $z_{rot(x,r),\Theta}(\ifout, 1, 0) = rot(x,r)(\ifout)= x(\ifout\oplus r) = z_{x,\Theta}(\ifout\oplus r, 1,0)$. Now for $l=1$, we have
\begin{align*}
q_{rot(x,r),\Theta}(\ifout,\cdot,1) &= \sum_{\iin\in[1],\icin\in[\wconv]}\Theta(\icin,\iin,\iout,l)\cdot z_{rot(x,r),\Theta}(\ifout \oplus(\icin-1),\iin,0)\\
&= \sum_{\iin\in[1],\icin\in[\wconv]}\Theta(\icin,\iin,\iout,l)\cdot z_{x,\Theta}((\ifout\oplus r) \oplus (\icin-1),\iin,0)\\
&= q_{x,\Theta}(\ifout\oplus r,\cdot,1) 
\end{align*}
The proof follows by noting that $G=\mathbf{1}_{\{q>0\}}$, and $z=q\cdot G$, and repeating the above argument for the layer $l=2,\ldots, \dc$.
\end{proof}


Proof of \Cref{lm:sumofproduct}
\begin{proof}
\begin{align}
\ip{\phi_{x_s,\Theta},\phi_{x_{s'},\Theta}}&=\sum_{p\in[P]}x_s(\I_0(p))x_{s'}(\I_0(p))A_{\Theta}(x_s,p)A_{\Theta}(x_{s'},p)\nn\\
&=\sum_{i=1}^{\din}x_s(i)x_{s'}(i)\Lambda_{\Theta}(i,x_s,x_{s'})\nn\\
&=\ip{x_s,x_{s'}}_{\Lambda_{\Theta}(\cdot,x_s,x_{s'})}
\end{align}
Owing to the symmetry in a fully connected network, we have $\Lambda(i,x_s,x_{s'})$ to be the same for all values of $i\in[\din]$. And since $H^{\text{lyr}}_{l,\Theta}(s,s')$ measure the number of gates in layer `$l$' that are active for both inputs $x_s$ and $x_{s'}$, the total number of paths active for both inputs is $\Pi_{l=1}^{(d-1)} H^{\text{lyr}}_{l,\Theta}(s,s')$.
\end{proof}

Proof of \Cref{lm:sumofproduct}
\begin{proof}
Proof is complete by noting that the NPF of the ResNet is a concatenation of the NPFs of the $2^b$ distinct sub-FC-DNNs within the ResNet architecture.
\end{proof}


Proof of \Cref{lm:cnnnpk}
\begin{proof}
For the CNN architecture considered in this paper, each bundle has exactly $\din$ number of paths, each one corresponding to a distinct input node. For a bundle $b_{\hat{p}}$, let $b_{\hat{p}}(i),i\in[\din]$ denote the path starting from input node $i$.
\begin{align*}
&\sum_{\hat{p}\in [\hat{P}]} \Bigg(\sum_{i,i'\in[\din]} x(i) x'(i') A_{\Theta}\left(x,b_{\hat{p}}(i)\right) A_{\Theta}\left(x',b_{\hat{p}}(i')\right) \Bigg)\\
=&\sum_{\hat{p}\in [\hat{P}]}\Bigg(\sum_{i\in[\din],i'=i\oplus r, r\in\{0,\ldots,\din-1\}} x(i) x'(i\oplus r) A_{\Theta}\left(x,b_{\hat{p}}(i)\right) A_{\Theta}\left(x',b_{\hat{p}}(i\oplus r)\right)\Bigg)\\
=&\sum_{\hat{p}\in [\hat{P}]}\Bigg(\sum_{i\in[\din], r\in\{0,\ldots,\din-1\}} x(i) rot(x',r)(i) A_{\Theta}\left(x,b_{\hat{p}}(i)\right) A_{\Theta}\left(rot(x',r),b_{\hat{p}}(i)\right)\Bigg)\\
=&\sum_{r=0}^{\din-1} \Bigg(\sum_{i\in[\din]} x(i) rot(x',r)(i) \sum_{\hat{p}\in [\hat{P}]}  A_{\Theta}\left(x,b_{\hat{p}}(i)\right) A_{\Theta}\left(rot(x',r),b_{\hat{p}}(i)\right)\Bigg)\\
=&\sum_{r=0}^{\din-1}\Bigg(\sum_{i\in[\din]} x(i) rot(x',r)(i) \Lambda_{\Theta}(i,x,rot(x',r))\Bigg)\\
=&\sum_{r=0}^{\din-1} \ip{x,rot(x',r)}_{\Lambda_{\Theta}(\cdot,x,rot(x',r))}
\end{align*}
\end{proof}

Proof of \Cref{th:main} follows in the same manner as the proof Theorem~$5.1$ of \citeauthor{ch2020neural} (2020).

\begin{comment}
\begin{lemma}\label{lm:dot}
Let $\varphi_{p,\Theta}$ be as in \Cref{def:npvgrad}, under Assumption~\ref{assmp:main}, for paths $p,p_1,p_2\in \P, p_1\neq p_2$, at initialisation we have (i) $\E{\ip{\varphi_{p_1,\Tv_0}, \varphi_{p_2,\Tv_0}}}= 0$, (ii) ${\ip{\varphi_{p,\Tv_0}, \varphi_{p,\Tv_0}}}= d\sigma^{2(d-1)}$.
\end{lemma}

\begin{proof}
\begin{align*}
\ip{\varphi_{p_1,\Tv_0}, \varphi_{p_2,\Tv_0}}= \sum_{\tv\in \Tv} \partial_{\tv}v_{\Tv_0}(p_1) \partial_{\tv}v_{\Tv_0}(p_2)
\end{align*}
Let $p\rsa(\cdot)$ denote the fact that path $p$ passes through $(\cdot)$, and let $p\bcancel\rsa(\cdot)$ denote the fact that path $p$ does not pass through $\bcancel\rsa$. Let $\tv\in\Tv$ be any weight such that $p\rsa \tv$, and w.l.o.g let $\tv$ belong to layer $l'\in[d]$. If either $p_1\bcancel{\rsa}\tv$ or $p_2\bcancel{\rsa}\tv$, then it follows that $\partial_{\tv} v_{\Tv_0}(p_1) \partial_{\tv} v_{\Tv_0}(p_2)=0$. In the case when $p_1,p_2\rsa\tv$, we have
\begin{align*}
&\E{\partial_{\tv}v_{\Tv_0}(p_1)\partial_{\tv}v_{\Tv_0}(p_2)}\\
&=\E{\underset{l\neq l'}{\underset{l=1}{\overset{d}{\Pi}}} \Bigg(\Tv_0(l, \I_{l-1}(p_1),\I_{l} (p_1))\Tv_0(l,\I_{l-1} (p_2),\I_{l}(p_2)) \Bigg)}\\
&=\underset{l\neq l'}{\underset{l=1}{\overset{d}{\Pi}}}\E{\Tv_0(l,\I_{l-1}(p_1),\I_{l}(p_1))\Tv_0(l,\I_{l-1}(p_2),\I_{l}(p_2))}
\end{align*}
where the $\E{\cdot}$ moved inside the product because at initialisation the weights (of different layers) are independent of each other.
Since $p_1\neq p_2$, in one of the layers $\tilde{l}\in[d-1],\tilde{l}\neq l'$ they do not pass through the same weight, i.e., $\Tv_0(\tilde{l},\I_{\tilde{l}-1}(p_1),\I_{\tilde{l}}(p_1))$ and $\Tv_0(\tilde{l},\I_{\tilde{l}-1}(p_2),\I_{\tilde{l}}(p_2))$ are distinct weights. Using this fact
\begin{align*}
&\E{\partial_{\tv}v_{\Tv_0}(p_1)\partial_{\tv}v_{\Tv_0}(p_2)}\\
&=\underset{l\neq l',\tilde{l}}{\underset{l=1}{\overset{d}{\Pi}}}\E{\Tv_0(l, \I_{l-1}(p_1),\I_l(p_1))\Tv_0(l,\I_{l-1}(p_2),\I_{l}(p_2))}\\
&=\E{\Tv_0(\tilde{l},\I_{\tilde{l}-1} (p_1),\I_{\tilde{l}}(p_1))}\E{\Tv_0(\tilde{l},\I_{\tilde{l}-1 }(p_2),\I_{\tilde{l}}(p_2))}\\
&=0
\end{align*}

The proof of (ii) is complete by noting that $\sum_{\tv\in\Tv} \partial_{\tv}v_{\Tv_0}(p) \partial_{\tv}v_{\Tv_0}(p)$ has $d$ non-zero terms for a single path $p$ and at initialisation we have 
\begin{align*}
&\partial_{\tv}v_{\Tv_0}(p) \partial_{\tv}v_{\Tv_0}(p) \\
&={\underset{l\neq l'}{\underset{l=1}{\overset{d}{\Pi}}} {\Tv_0}^2(l,\I_{l-1}(p),\I_{l}(p))}\\
&=\sigma^{2(d-1)}
\end{align*}
\end{proof}

\textbf{Detailed version of \Cref{th:main} with proof.}
\begin{theorem}\label{th:mainrefined}
Under \Cref{assmp:main}, and $\frac{4d}{w^2}<1$ it follows that
 \begin{align*}
\E{K_{\Tdgn_0}}&=d\cdot\sigma^{2(d-1)} H_{\text{FNPF}}\\
Var\left[K_{\Tdgn_0}(s,s')\right]&\leq O\left(d^2_{in}\sigma^{4(d-1)}\max\{d^2w^{2(d-2)+1}, d^3w^{2(d-2)}\}\right)
\end{align*}
\end{theorem}

\begin{proof}
We have 
\begin{align*}
\E{K_{\Tdgn_0}}&=\E{\Phi^\top_{\text{FNPF}} \V_{\Tv_0} \Phi_{\text{FNPF}}}\\
&=\E{\Phi^\top_{\text{FNPF}} (\nabla_{\Tv}v_{\Tv_0})^\top (\nabla_{\Tv}v_{\Tv_0}) \Phi_{\text{FNPF}}}\\
&=\Phi^\top_{\text{FNPF}} \E{(\nabla_{\Tv}v_{\Tv_0})^\top (\nabla_{\Tv}v_{\Tv_0})}\Phi_{\text{FNPF}}\\
&\stackrel{(a)}=d\cdot\sigma^{2(d-1)} \Phi^\top_{\text{FNPF}}\Phi_{\text{FNPF}}\\
&=d\cdot\sigma^{2(d-1)} H_{\text{FNPF}}
\end{align*}
where, $(a)$ follows from \Cref{lm:dot}.

We now turn to the variance calculation. The idea is that we expand  $Var\left[K_0(s,s')\right]=\E{K_0(s,s')^2} -\E{K_0(s,s')}^2$ and identify the terms which cancel due to subtraction and then bound the rest of the terms.

\textbf{Notation:} In what follows, we let $K_0$ to denote $K_{\Tdgn_0}$ and drop superscript V from $\Tv_0$, and subscript $\Tv_0$ from $v_{\Tv_0}$. Further, we assume that the weights can be enumerated as $\theta(1),\ldots, \theta(d_{net})$. We also denote $p\rsa (\cdot)$ to denote the fact that path $p$ passes through $(\cdot)$ and $p\bcancel{\rsa}(\cdot)$ to denote the fact that path $p$ does not pass through $(\cdot)$. We use a shortcut notation $A(s,p)$ instead of $A(x_s,p)$. In what follows, we let $x\in\R^{\din\times n}$ to be the data matrix.

 Let $\theta(m),m\in[d_{net}]$ belong to layer $l'(m)$, then 
\begin{align}\label{eq:kexpect} 
&\E{K_0(s,s')}\nn\\
&=\sum_{m=1}^{d_{net}}\E{\left(\sum_{p_1 \in[P]}x(\I_0(p_1),s)A_0(s,p_1)\frac{\partial v_0(p_1)}{\partial \theta(m)}\right)\left(\sum_{p_2\in[P]}x(\I_0(p_2),s)A_0(s',p_2)\frac{\partial v_0(p_2)}{\partial \theta(m)}\right)}\nn\\
&=\sum_{m=1}^{d_{net}}\E{\sum_{p_1,p_2\in[P]}x(\I_0(p_1),s)A_0(s,p_1)\frac{\partial v_0(p_1)}{\partial \theta(m)}x(\I_0(p_2),s')A_0(s',p_2)\frac{\partial v_0(p_2)}{\partial \theta(m)}}\nn\\
&\stackrel{(a)}=\sum_{m=1}^{d_{net}}\underset{p_1,p_2\rsa\theta(m)}{\sum_{p_1,p_2\in[P]}}x(\I_0(p_1),s)A_0(s,p_1)x(\I_0(p_2),s')A_0(s',p_2) \mathbb{E}\Bigg[\underset{l\neq l'(m)}{\underset{l=1}{\overset{d-1}{\Pi}}} \Tb_0(l,\I_{l-1}(p_1),\I_{l}(p_1)) \nn\\&\Tb_0(l,\I_{l-1}(p_2),\I_{l}(p_2))\Bigg]\nn\\
&\stackrel{(b)}=\sum_{m=1}^{d_{net}}\underset{p_1,p_2\rsa\theta(m)}{\sum_{p_1,p_2\in[P]}}x(\I_0(p_1),s)A_0(s,p_1)x(\I_0(p_2),s')A_0(s',p_2) \underset{l\neq l'(m)}{\underset{l=1}{\overset{d-1}{\Pi}}} \mathbb{E}\Bigg[ \Tb_0(l,\I_{l-1}(p_1),\I_{l}(p_1))\nn\\
& \Tb_0(l,\I_{l-1}(p_2),\I_{l}(p_2))\Bigg]
\end{align}
where $(a)$ follows from the fact that for $p\bcancel{\rsa}\theta(m)$, $\frac{\partial v_0(p)}{\partial \theta(m)}=0$, and $(b)$ follows from the fact that at initialisation the layer weights are independent of each other. Note that the right hand side of \eqref{eq:kexpect} only terms with $p_1=p_2$ will survive the expectation.

In the following expression in \eqref{eq:kexpectsquare}, note that only terms of the form $p_1=p_2$ and $p_3=p_4$ are non-zero.

\begin{align*}
&\E{K_0(s,s')}^2=\\
&\Bigg(\sum_{m=1}^{d_{net}}\underset{p_1,p_2\rsa\theta(m)}{\sum_{p_1,p_2\in[P]}}x(\I_0(p_1),s)A_0(s,p_1)x(\I_0(p_2),s')A_0(s',p_2) \underset{l\neq l'(m)}{\underset{l=1}{\overset{d-1}{\Pi}}} \mathbb{E}\Big[\Tb_0(l,\I_{l-1}(p_1),\I_{l}(p_1)) &\\ 
&\Tb_0(l,\I_{l-1}(p_2),\I_{l}(p_2))\Big]\Bigg)\times\\
&\Bigg(\sum_{m'=1}^{d_{net}}\underset{p_3,p_4\rsa\theta(m')}{\sum_{p_3,p_4\in[P]}}x(\I_0(p_3),s)A_0(s,p_3)x(\I_0(p_4),s')A_0(s',p_4) \underset{l\neq l'(m')}{\underset{l=1}{\overset{d-1}{\Pi}}} \mathbb{E}\Big[\Tb_0(l,\I_{l-1}(p_3),\I_{l}(p_3)) &\\
&\Tb_0(l,\I_{l-1}(p_4),\I_{l}(p_4))\Big]\Bigg)\\
\end{align*}
\begin{align}\label{eq:kexpectsquare}
&\E{K_0(s,s')}^2=\nn\\
&\sum_{m,m'=1}^{d_{net}}\underset{p_3,p_4\rsa\theta(m')}{\underset{p_1,p_2\rsa\theta(m)}{\sum_{p_1,p_2,p_3,p_4\in[P]}}}\Bigg[\bigg(x(\I_0(p_1),s)A_0(s,p_1)x(\I_0(p_2),s')A_0(s',p_2)x(\I_0(p_3),s)\nn\\ 
&A_0(s,p_3)x(\I_0(p_4),s')A_0(s',p_4)\bigg)\times \bigg( \underset{l\neq l'(m)} {\underset{l\neq l'(m')}{\underset{l=1}{\overset{d-1}{\Pi}}}} \E{\Tb_0(l,\I_{l-1}(p_1),\I_{l}(p_1)) \Tb_0(l,\I_{l-1}(p_2),\I_{l}(p_2))}\nn\\
&\E{\Tb_0(l,\I_{l-1}(p_3),\I_{l}(p_3)) \Tb_0(l,\I_{l-1}(p_4),\I_{l}(p_4))} \bigg)\times\nn\\
&\bigg( \E{\Tb_0(l,\I_{l'(m')-1}(p_1),\I_{l'(m')}(p_1)) \Tb_0(l,\I_{l'(m')-1}(p_2),\I_{l'(m')}(p_2))}\bigg)\times\nn\\
&\bigg(\E{\Tb_0(l,\I_{l'(m)-1}(p_3),\I_{l'(m)}(p_3)) \Tb_0(l,\I_{l'(m)-1}(p_4),\I_{l'(m)}(p_4))} \bigg)\Bigg]
\end{align}
In the expression in \eqref{eq:ksquareexpect}, paths $p_1,p_2,p_3,p_4$ do not have constraints, and can be distinct.
\begin{align}\label{eq:ksquareexpect}
&\E{K^2_0(s,s')}=\nn\\
&\sum_{m,m'=1}^{d_{net}}\underset{p_3,p_4\rsa\theta(m')}{\underset{p_1,p_2\rsa\theta(m)}{\sum_{p_1,p_2,p_3,p_4\in[P]}}}\Bigg[\bigg(x(\I_0(p_1),s)A_0(s,p_1)x(\I_0(p_2),s')A_0(s',p_2)x(\I_0(p_3),s)\nn\\&
A_0(s,p_3)x(\I_0(p_4),s')A_0(s',p_4)\bigg)\times\bigg( \underset{l\neq l'(m)} {\underset{l\neq l'(m')}{\underset{l=1}{\overset{d-1}{\Pi}}}} \mathbb{E}[\Tb_0(l,\I_{l-1}(p_1),\I_{l}(p_1)) \Tb_0(l,\I_{l-1}(p_2),\I_{l}(p_2))\nn\\&
\Tb_0(l,\I_{l-1}(p_3),\I_{l}(p_3)) \Tb_0(l,\I_{l-1}(p_4),\I_{l}(p_4))] \bigg)\times\nn\\
&\bigg( \E{\Tb_0(l,\I_{l'(m')-1}(p_1),\I_{l'(m')}(p_1)) \Tb_0(l,\I_{l'(m')-1}(p_2),\I_{l'(m')}(p_2))}\bigg)\times\nn\\
&\bigg(\E{\Tb_0(l,\I_{l'(m)-1}(p_3),\I_{l'(m)}(p_3)) \Tb_0(l,\I_{l'(m)-1}(p_4),\I_{l'(m)}(p_4))} \bigg)\Bigg]
\end{align}

We now state the following facts/observations.

$\bullet$ \emph{Fact 1:} Any term that survives the expectation (i.e., does not become $0$) and participates in \eqref{eq:ksquareexpect} is of the form $\sigma^{4(d-1)}\big(x(\I_0(p_1),s)A_0(s,p_1)x(\I_0(p_2),s')A_0(s',p_2)x(\I_0(p_3),s)A_0(s,p_3)x(\I_0(p_4),s')A_0(s',p_4)\big)$, where $p_1,p_2,p_3,p_4$ are free variables. Any term that survives the expectation (i.e., does not become $0$) and participates in participates in \eqref{eq:kexpectsquare} is of the form $\sigma^{4(d-1)}\big(x(\I_0(p_1),s)A_0(s,p_1)x(\I_0(p_2),s')A_0(s',p_2)x(\I_0(p_3),s)A_0(s,p_3)x(\I_0(p_4),s')A_0(s',p_4)\big)$, where $p_1=p_2,p_3=p_4$.

$\bullet$ \emph{Fact 2:} The number of paths through a particular weight $\theta(m)$ in one of the middle layers is $\din w^{d-3}$. The number of paths through a particular weight $\theta(m)$ in the first layer is $w^{d-2}$ . The number of paths through a particular weight $\theta(m)$ in the last layer is $\din w^{d-2}$ .


$\bullet$ \emph{Fact 3:} Let $\P'$ be an arbitrary set of paths constrained to pass through some set of weights. Let $\P''$ be the set of paths obtained by adding an additional constraint that the paths also should pass through a particular weight say $\theta(m)$. Now, if $\theta(m)$ belongs to :

$1.$ a middle layer, then $|\P''|=\frac{|\P'|}{w^2}$.

$2.$ the first layer, then $|\P''|=\frac{|\P'|}{\din w}$.

$3.$  the last layer, then $|\P''|=\frac{|\P'|}{w}$.

$\bullet$ \emph{Fact 4:} For any $p_1,p_2,p_3,p_4$ combination that survives the expectation in \eqref{eq:ksquareexpect} can be written as 

\begin{align*}
&\bigg(x(\I_0(p_1),s)A_0(s,p_1)x(\I_0(p_2),s')A_0(s',p_2)x(\I_0(p_3),s)\\
&A_0(s,p_3)x(\I_0(p_4),s')A_0(s',p_4)\bigg)\times\\
&\bigg( \underset{l\neq l'(m)} {\underset{l\neq l'(m')}{\underset{l=1}{\overset{d-1}{\Pi}}}} \mathbb{E}[\Tb_0(l,\I_{l-1}(p_1),\I_{l}(p_1)) \Tb_0(l,\I_{l-1}(p_2),\I_{l}(p_2))\nn\\
&\Tb_0(l,\I_{l-1}(p_3),\I_{l}(p_3)) \Tb_0(l,\I_{l-1}(p_4),\I_{l}(p_4))] \bigg)\times\nn\\
&\bigg( \E{\Tb_0(l,\I_{l'(m')-1}(p_1),\I_{l'(m')}(p_1)) \Tb_0(l,\I_{l'(m')-1}(p_2),\I_{l'(m')}(p_2))}\bigg)\times\nn\\
&\bigg(\E{\Tb_0(l,\I_{l'(m)-1}(p_3),\I_{l'(m)}(p_3)) \Tb_0(l,\I_{l'(m)-1}(p_4),\I_{l'(m)}(p_4))} \bigg)\nn\\
%&=\\
%&\bigg( \underset{l\neq l'(m)} {\underset{l\neq l'(m')}{\underset{l=1}{\overset{d-1}{\Pi}}}} \Tb^2_0(l,\I_{l-1}(\rho_a),\I_{l}(\rho_a)) \Tb^2_0(l,\I_{l-1}(\rho_b),\I_{l}(\rho_b)) \bigg)\times \bigg( \Tb^2_0(l,\I_{l'(m')-1}(\rho_a),\I_{l'(m')}(\rho_a))\bigg)\nn\\
%&\times \bigg({\Tb^2_0(l,\I_{l'(m)-1}(\rho_b),\I_{l'(m)}(\rho_b))} \bigg),
\end{align*}

where $\rho_a\rsa \theta(m)$ and $\rho_b\rsa \theta(m')$ are what we call as \emph{base} (case) paths. 


$\bullet$ \emph{Fact 5:} For any given base paths $\rho_a$ and $\rho_b$ there could be multiple assignments possible for $p_1,p_2,p_3,p_4$.

$\bullet$ \emph{Fact 6:}  Terms in \eqref{eq:ksquareexpect}, wherein, the base case is generated as $p_1=p_2=\rho_a$ and $p_3=p_4=\rho_b$ (or $p_1=p_2=\rho_b$ and $p_3=p_4=\rho_a$), get cancelled with the corresponding terms in \eqref{eq:kexpectsquare}.

$\bullet$ \emph{Fact 7:}  When the bases paths $\rho_a$ and $\rho_b$ do not intersect (i.e., do not pass through the same weight in any one of the layers), the only possible assignment is $p_1=p_2=\rho_a$ and $p_3=p_4=\rho_b$ (or $p_1=p_2=\rho_b$ and $p_3=p_4=\rho_a$), and such terms are common in \eqref{eq:ksquareexpect} and \eqref{eq:kexpectsquare}, and hence do not show up in the variance term.


$\bullet$ \emph{Fact 7:} Let base paths $\rho_a$ and $\rho_b$ intersect/cross at layer $l_1, \ldots, l_k, k \in [d-1]$, and let $\rho_a=(\rho_a(1),\ldots,\rho_a(k+1))$ where $\rho_a(1)$ is a sub-path string from layer $1$ to $l_1$, and $\rho_a(2)$ is the sub-path string from layer $l_1+1$ to $l_2$ and so on, and $\rho_a(k+1)$ is the sub-path string from layer $l_k+1$ to the output node. Then the set of paths that can occur in $\E{K_0(s,s')^2}$ are of the form:
\begin{enumerate}
\item $p_1=p_2=\rho_a, p_3=p_4=\rho_b$ (or $p_1=p_2=\rho_b, p_3=p_4=\rho_a$) which get cancelled in the $\E{K_0(s,s')}^2$ term.
\item $p_1=\rho_a$, $p_3=\rho_b$, $p_2=(\rho_b(1),\rho_a(2),\rho_a(3),\ldots,\rho_a(k+1))$, $p_4=(\rho_a(1),\rho_b(2),\rho_b(3),\ldots,\rho_b(k+1))$, which are obtained by \emph{splicing} the base paths in various combinations. Note that for such spliced paths $p_1\neq p_2$ and $p_3\neq p_4$ and hence do not occur in the expression for $\E{K_0(s,s')}^2$ in \eqref{eq:kexpectsquare}.
\end{enumerate}


$\bullet$ \emph{Fact 8:} For $k$ crossings of the base paths there are $4^{k+1}$ splicings possible, and those many terms are extra in the $\E{K_0(s,s')^2}$ expression in \eqref{eq:ksquareexpect}, when compared to the $\E{K_0(s,s')}^2$ expression in \eqref{eq:kexpectsquare}. 

\textbf{Upper Bound:} We now enumerate various possible crossings of the base paths, and calculate an upper bound for the magnitude of the contribution of `spliced' terms to the variance term using the \emph{Fact 1} to \emph{Fact 8}. In short, we find an upper bound for the those terms that do not get cancelled in the variance calculation. Further, without loss of generality we drop $x(\I_0(p))$ and $A(\cdot,\cdot)$ terms in this upper bound calculation.

\begin{tabular}{|c|c|}\hline
Architecture& Constant \\\hline
FC-DNN&	$\cfc=\din^2 d^2\sigma^{4(d-1)}$\\\hline
CNN& $\ccnn=\din^2 d^2\sigma^{4(d-1)}\wconv$\\\hline
ResNet & $\cres=\din^2\left(\sum_{i=0}^b \binom{b}{i} (i+2) d_{\text{block}}\sigma^{2((i+2) d_{\text{block}}-1)} \right)^2$\\\hline
\end{tabular}
\FloatBarrier
\begin{table}[h]
\begin{tabular}{|c|l|c|c|c|}\hline
Case &Crossing					&FC-DNN 					&CNN 									&ResNet\\\hline
1& $1$ F or $1$ L				&$\frac{2\cdot6^2\cfc}{w}$  		&$\frac{2\cdot6^2\ccnn}{w}$ 			&$\frac{2\cdot6^2\cres}{w}$\\\hline
2& $1$ M					&$\frac{36 d\cfc }{w^2}$  			&$\frac{36 d\ccnn}{w^2}$ 						&$\frac{36 d \cres}{w^2}$\\\hline
3& $1$ F and $1$ L				&$\frac{6^3\cfc}{\din w^2}$  		&$\frac{6^3\ccnn}{\din\wconv w^2}$ 	&$\frac{6^3\cres}{\din w^2}$\\\hline
4& $1$ F or $1$ L and $1$ M		&$\left(\frac{6d}{w^2}\right)\left(\frac{2\cdot6^2\cfc}{w}\right)$ &$\left(\frac{6d}{w^2}\right)\left(\frac{2\cdot6^2\ccnn}{w}\right)$  &$\left(\frac{6d}{w^2}\right)\left(\frac{2\cdot6^2\cres}{w}\right)$\\\hline
5& $2$ M 					&$\left(\frac{6d}{w^2}\right)\left(\frac{36 d\cfc }{w^2}\right)$  			&$\left(\frac{6d}{w^2}\right)\left(\frac{36 d\ccnn}{w^2}\right)$ 						&$\left(\frac{6d}{w^2}\right)\left(\frac{36 d \cres}{w^2}\right)$\\\hline
6& $1$ F or $1$ L and $2$ M		&$\left(\frac{6d}{w^2}\right)^2\left(\frac{2\cdot6^2\cfc}{w}\right)$ &$\left(\frac{6d}{w^2}\right)^2\left(\frac{2\cdot6^2\ccnn}{w}\right)$  &$\left(\frac{6d}{w^2}\right)^2\left(\frac{2\cdot6^2\cres}{w}\right)$\\\hline
7& $1$ F and $1$ L and $1$ M 		&$\left(\frac{6d}{w^2}\right)\left(\frac{6^3\cfc}{\din w^2}\right)$  		&$\left(\frac{6d}{w^2}\right)\left(\frac{6^3\ccnn}{\din \wconv w^2}\right)$ 	&$\left(\frac{6d}{w^2}\right)\left(\frac{6^3\cres}{\din w^2}\right)$\\\hline
8& $3$M 					&$\left(\frac{6d}{w^2}\right)^2\left(\frac{36 d\cfc }{w^2}\right)$  			&$\left(\frac{6d}{w^2}\right)^2\left(\frac{36 d\ccnn}{w^2}\right)$ 						&$\left(\frac{6d}{w^2}\right)^2\left(\frac{36 d \cres}{w^2}\right)$\\\hline
Total & Cases 1+4+6+$\ldots$ &$\frac{\cfc}{w}\left(\frac{\cfc'}{1-6dw^{-2}}\right)$ &$\frac{\ccnn}{w}\left(\frac{\ccnn'}{1-6dw^{-2}}\right)$& $\frac{\cres'}{w}\left(\frac{\cres}{1-6dw^{-2}}\right)$\\\hline
Total & Cases 3+7+$\ldots$ &$\frac{d\cfc}{w^2}\left(\frac{\cfc''}{1-6dw^{-2}}\right)$ &$\frac{d\ccnn}{w^2}\left(\frac{\ccnn''}{1-6dw^{-2}}\right)$& $\frac{d\cres}{w^2}\left(\frac{\cres''}{1-6dw^{-2}}\right)$\\\hline
Total & Cases 2+4+6+$\ldots$ &$\frac{\cfc}{w^2}\left(\frac{\cfc'''}{1-6dw^{-2}}\right)$ &$\frac{\ccnn}{w^2}\left(\frac{\ccnn'''}{1-6dw^{-2}}\right)$& $\frac{\cres}{w^2}\left(\frac{\cres'''}{1-6dw^{-2}}\right)$\\\hline
\end{tabular}
\caption{Here `F', `M' and `L' stand for \emph{first, middle} and \emph{last} layers respectively, where middle layer is any layer that is not the first or the last. $1$ F or $1$ L means, $k=1$ crossing of the base paths either in the first or the last layer. $1$ F and $1$ L means $k=2$ crossings of the base paths one in the first layer and the other in the last layer. Similarly $3$ M means, $k=3$ crossings of the base paths, all in the middle (i.e., intermediate) layers.} 
\end{table}
The cases can be extended in a similar way, increasing the number of crossings.  Now, assuming $\frac{6d}{w^2}<1$, the bounds in the various terms can be lumped together as below:

Putting together we have the variance to be bounded by 
\begin{align*}
\cfc\cdot\max\left\{\cfc^{(1)}\cdot\left(\frac{1}{w}\right),\cfc^{(2)}\cdot\left(\frac{d}{w^2}\right)\right\}, \\
\ccnn\cdot\max\left\{\ccnn^{(1)}\cdot\left(\frac{1}{w}\right),\ccnn^{(2)}\cdot\left(\frac{d}{w^2}\right)\right\}, \\
\cres\cdot\max\left\{\cres^{(1)}\cdot\left(\frac{1}{w}\right),\cres^{(2)}\cdot\left(\frac{d}{w^2}\right)\right\}, 
\end{align*}
respectively for FC-DNN, CNN and ResNet. Here, $\cfc^{(i)},\ccnn^{(i)},\cres^{(i)},i=1,2$ are positive constants.
\end{proof}
\end{comment}
\end{appendix}

\end{document}
